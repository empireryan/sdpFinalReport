\section{Background}
In order to meet the challenge of building next-generation power conversion technologies capable of handling highly variable sources while simultaneously mitigating the problem of noise injection into the power grid, we must first understand the current landscape of of power converters, analyze their strengths and weaknesses, and then assess how we might improve upon their implementations by leveraging today's inexpensive real-time microcontrollers which have been tuned for power today's demanding power conversion applications. To this end, we will undertake a brief overview of the pulse-width modulation technique, the applicability of hybrid control algorithms to the problem of power conversion, and finally, we will briefly review the additional constraints that power conversion in the context of solar energy places on our design. 

\subsection{The Problems with Solar}
With the cost of photovoltaics rapidly decreasing, we have seen a rush toward the adoption of solar micro-grids which seek to exploit the most abundant source of power known to mankind - the sun. However, our heliocentric conundrum - namely the fact that the earth orbits the sun - dictates that most places on earth receive time-dependent quantities of solar irradiance over the course of any given day. This high variability in the context of power conversion implies major challenges to our modern power grid which guarantees nearly continuous up-time. Additionally, today's power conversion technologies are not robust to highly variable input sources like solar power. Some other challenges we face when converting DC solar power to AC power which is usable in our power grid is the problem of shading or partial irradiance of solar arrays, and the non-linear nature of the photovoltaics themselves. In particular, this non-linear behavior necessitates the implementation of maximum power point tracking or MPPT algorithms which comprehend this phenomena and harvest the most power from solar sources which typically have a finite lifespan. This is critical for obtaining the maximum benefit from the non-trivial investment required to operate solar arrays today. 

\subsection{The PWM approach}
Pulse-width modulation (PWM), or bang-bang techniques, are a popular means for taking a fixed input voltage and varying the effective power seen at the output of some system using some form of switching. The switch in this case can take the form of a relay, but more commonly in power applications, takes the form of a field-effect transistor (MOSFET) or insulated gate bipolar transistor (IGBT). The effective voltage or current is varied by varying the duration of on-time of the switch. One can think of this as int terms of a common household faucet with only two states, either fully on or off. If we were able to modulate the amount of time that the water flowed from the faucet, clearly we would be able to choose the flow-rate of the water from the two extrema - on or off - to an arbitrary level of precision depending on how fast we were able to turn the faucet on or off. The scenario put forth in the 'kitchen sink' analogy is analogous to the situation we face in a modern power inverter. In this case, we are faced with the challenge of taking a typically fixed input voltage and switching it in such a way that we achieve a close approximation to a sinusoidal output voltage. Given the clear explanation of how PWM techniques can vary from fully on to fully off in the description above, it is easy to see why the PWM approach has become the defacto standard for power inverters. Additionally, today's microcontrollers have extremely sophisticated peripheral modules built specifically for very fast and PWM signal generation with resolution down to the nanosecond level becoming quite common. 

Although the PWM method for signal modulation has the clear advantages of widespread adoption and ease of implementation - it is quite common - with the widespread adoption of renewable energy in the form of highly local and decentralized micro-grids, we are increasingly faced with the problem of introducing increasing amounts of high-order noise into the power grid. Further, the typical proportional, integral, derivative, or PID methods for control suffer from a ringing phenomena in the presence of input disturbances which are typical of renewable energy applications. In light of these issues with the PWM technique for power conversion, we seek to explore alternate strategies for switching that might alleviate the vulnerability to  input disturbances and the problem of an unintentionally rich spectrum at the output of PWM power converters. 

\subsection{Hybrid Systems and Hybrid Control}
The emerging field of study known as hybrid dynamical control seeks to study systems that exhibit both continuous and discrete-time dynamics. The system in question, namely the canonical power inverter topology that is the focus of this research, is an excellent candidate for study under the lens of hybrid control due to the continuous evolution of a linear oscillator - our output filter - and the discrete time dynamics of the switch - in our case, some type of transistor. 

We seek to understand how hybrid control might alleviate some of the problems with PWM techniques, a few of which have been outlined above. While a detailed explanation of the hybrid control algorithm is saved for later on in this paper, in a nutshell it has been found that with relatively simple physical models of our system, and the availability of discrete time switching, that we can generate outputs that closely resemble the desired sinusoidal outputs with less switching noise, and with a robust response to highly variable input voltages. For these reasons, we focus this research on the physical realization of such a hybrid system. One of the goals of this paper is to describe the algorithm designed by Jun Chai and Dr. Ricardo Sanfelice in more detail, and outline some of the challenges associated with realizing this implementation on a real world system \cite{91}.